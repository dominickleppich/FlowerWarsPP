\documentclass[12pt]{article}

\usepackage[a4paper,top=1in, bottom=1in, left=1in, right=1in]{geometry}

\usepackage[utf8]{inputenc} %Eingabecodierung
\usepackage[ngerman]{babel} %Deutsche Sprache
\usepackage[T1]{fontenc} %T1 Zeichenkodierung, fuer automatische Zeilenumbrueche nach Umlauten
\usepackage[markup=nocolor]{changes}

\usepackage{lmodern} %Moderner Zeichensatz
\usepackage{textcomp}
\usepackage{hyperref}

\usepackage{graphicx} %Grafiken laden
\usepackage{caption}

\usepackage{tikz,pgf} % TikZ Grafiken
\usepackage{pdfpages}
\usepackage{pgffor}
\usetikzlibrary{arrows}
\usetikzlibrary{calc}

\usepackage{enumitem} %Manuelle Numerierungsoptionen
\usepackage{float} % Figuren an der korrekten Quellcode Position anzeigen
\usepackage{pdflscape} % Querlegen

\usepackage{listings}
\lstset{
	basicstyle=\scriptsize\ttfamily,
	breakatwhitespace=false,
	commentstyle=\color{darkgray},
	frame=lines,
	framexleftmargin=1.5em,
	language=Java,
	numbers=left,
	numbersep=10pt,
	numberstyle=\tiny,
	rulecolor=\color{black},
	showstringspaces=true,
	tabsize=4,
	xleftmargin=1.5em
}

\usepackage{amsthm} %Mathematiksatz-Umgebung
\usepackage{amssymb} %Mathematische Symbole
\usepackage{amsmath}

\theoremstyle{plain}
\newtheorem{theorem}[equation]{Theorem}
\newtheorem{satz}[equation]{Satz}
\newtheorem{lemma}[equation]{Lemma}
\newtheorem{korollar}[equation]{Korollar}

\theoremstyle{definition}
\newtheorem{definition}[equation]{Definition}
\newtheorem{aufgabe}{Aufgabe}
\newtheorem*{behauptung}{Behauptung}
\newtheorem{uebung}{Übung}
\newtheorem*{loesung}{Lösung}

\theoremstyle{remark}
\newtheorem{beispiel}[equation]{Beispiel}
\newtheorem{hinweis}{Hinweis}

\newcommand{\ggT}{\textrm{ggT}}
\newcommand{\kgV}{\textrm{kgV}}
\newcommand{\code}[1]{\texttt{#1}}
\newcommand{\examplegraphic}[3]{
\begin{figure}[ht]
\begin{center}
\includegraphics[scale=#2]{#1} \\
\smallskip
{\footnotesize #3}
\end{center}
\end{figure}
}

\setlength{\parindent}{0cm}
\setlength{\parskip}\medskipamount

\begin{document}

\textbf{Allgemeines Programmierpraktikum}\hfill \textbf{Sommersemester 2018} \\
Dr. Henrik Brosenne\hfill Georg-August-Universität Göttingen \\
Dominick Leppich\hfill Institut für Informatik\\
\medskip
\hrule
\bigskip
\begin{center}
{\Large \textbf{Projekt}}
\end{center}
\bigskip
\textbf{Abgabe bis 15.07.2018, 23:59 Uhr}. 

\textbf{Prüfungen in der Zeit vom 07.08. bis 16.08.2018}.

\section*{Organisation}
\begin{enumerate}
\item Wenn Sie an der Prüfung zum Modul \emph{B.Inf.1802: Programmierpraktikum} teilnehmen möchten, müssen Sie sich in \textbf{FlexNow} anmelden.

\textbf{An- und Abmeldefrist} für die Prüfung ist der \textbf{15.07.2018}.
\item Bilden Sie Projektgruppen mit vier Teilnehmern, größere Gruppen müssen ausdrücklich genehmigt werden und bekommen zusätzliche Aufgaben.
\item Vereinbaren Sie mit einem Tutor, der Ihr Projekt betreuen soll, einen Termin für ein regelmäßiges Treffen.
\item Bestimmen Sie einen Projektleiter. Zu den Aufgaben des Projektleiters gehört es die Entwicklung des Projekts als Ganzes zu steuern.
\begin{itemize}
\item Werden alle Anforderungen erfüllt?
\item Sind Sprache und Stil der Dokumentation einheitlich?
\item Ist eine überarbeitete Klasse von einem weiteren Projektmitglied kontrolliert worden?
\item \dots
\end{itemize}
\item Geben Sie der Projektgruppe einen aussagefähigen Namen ungleich \code{FlowerWarsPP}.
\item Der Projektleiter meldet die Projektgruppe an, per E-Mail an \\\href{mailto:brosenne@informatik.uni-goettingen.de}{\code{brosenne@informatik.uni-goettingen.de}} mit Betreff \code{FlowerWarsPP}.

In der E-Mail müssen Projektname, sowie die Namen und Matrikelnummern der Teilnehmer übermittelt werden.
\item Der Projektleiter reserviert einen Prüfungstermin, in Absprache mit den Mitgliedern der Projektgruppe und dem \textbf{Tutor}, unter \textit{Terminvergabe\textrightarrow APP-Prüfung} in der Stud.IP-Veranstaltung \textit{Allgemeines Programmierpraktikum} (oder im Profil des Stud.IP-Benutzers \textit{Rechnerübung Informatik}).

Bei der Reservierung des Prüfungstermins muss der Projektname angegeben werden.
\end{enumerate}
 \newpage
\section*{Git}

% Das ist das alte FusionForge Zeugs und muss an Git angepasst werden...
\begin{enumerate}
\item Melden Sie sich unter \code{https://fusionforge.informatik.uni-goettingen.de/} als Benutzer an.
\item Bestimmen Sie einen aus Ihrer Gruppe zum Fusionforge-Administrator, der dann ein neues Fusionforge-Projekt, unter dem Namen der Projektgruppe, für Ihre Gruppe anlegt.

Folgende Einstellungen sind vorzunehmen.
\begin{enumerate}[label=\alph*)]
\item Als Source Code Management (SCM) wählen Sie Subversion (SVN).
\item Alle Gruppenmitglieder registrieren sich im Projekt (oder werden vom Administrator registriert).
\item Das SVN-Verzeichnis ist nicht öffentlich lesbar, also nur für Mitarbeiter des Projektes zugänglich.
\item Registrieren Sie auch Ihren Tutor als Projektmitarbeiter.
\end{enumerate}
\end{enumerate} \newpage
\section*{Prüfung}
Nach der Abgabe wird das Projekt als Ganzes bewertet.

Während der Prüfung stellt jeder Teilnehmer den Teil des Projektes vor, für dessen Implementation er verantwortlich ist. Besonders interessant sind die aufgetretenen Probleme und deren Lösungen.

Neben der korrekten Umsetzung der Projektanforderungen wird gut lesbarer und strukturierter Quellcode erwartet. Es sollten die Grundlagen des objektorientierten Programmentwurfs (z.B. Kapselung, Vererbung, Polymorphismus) berücksichtigt und die Möglichkeiten von Java (z.B. \emph{Java Collections Framework}) ausgenutzt werden.

Jedem Teilnehmer werden Fragen zum Projekt, sowie zu Java, JavaDoc, Git und Ant, im Rahmen von Vorlesung und Übung, gestellt. Daraus ergibt sich für jeden Teilnehmer eine Tendenz (z.B. etwas schwächer als das Projekt insgesamt) und letztlich eine individuelle Note.
\section*{FlowerWarsPP}
\begin{quote}
Die Gebrüder Gartenpfleger Torsten und Torben streiten sich um Tamara und möchten sie durch das Herrichten einer hübschen Gartenlandschaft beeindrucken. Leider müssen sich die beiden die Wiese hierfür teilen. Wer schafft es die meisten Gärten zu bepflanzen und diese zu verzieren indem sie durch Gräben miteinander verbunden werden?
\end{quote}

Alle Informationen zu dem Spiel finden Sie auf der GitLab Seite des Projektes.

\href{https://gitlab.gwdg.de/app/flowerwarspp}{\code{https://gitlab.gwdg.de/app/flowerwarspp}}

\bigskip
\bigskip

Auf dieser Seite sind folgende Dinge zu finden:
\begin{itemize}
\item Spielregeln
\item Alle vorgegebenen Klassen des Projektes
\item Hilfestellungen zur Implementierung
\item Eine Strategie für einen einfachen Computerspieler
\item Eine Anleitung zur Nutzung des \code{ArgumentParser}
\item Eine Anleitung zur Nutzung der automatisierten Tests
\item Eine Anleitung zur Einbindung des grafischen Board-Viewers
\end{itemize}
 \newpage
\section*{Implementierung}
Realisieren Sie \textbf{FlowerWarsPP} als Computerspiel in Java. Kommentieren Sie den Quellcode ausführlich. Verwenden Sie JavaDoc für das \emph{Application Programming Interface (API)} und kommentieren Sie sonst wie üblich.

\begin{enumerate}
% Setup
\item Alle Klassen und Schnittstellen gehören zu einem Package, das mit \code{flowerwarspp} beginnt.

Die vorgegebenen Klassen und Schnittstellen des Package \code{flowerwarspp.preset} dürfen nicht verändert werden. 

Es ist allerdings erlaubt folgende Klassen zu erweitern (nähere Informationen sind dem entsprechenden Abschnitt der Projektbeschreibung zu entnehmen):
\begin{description}
\item[\code{Viewer}] Darf um weitere Funktionen erweitert werden. Vorhandene Funktionen dürfen nicht geändert werden.
\item[\code{ArgumentParser}] Darf um weitere Parameter ergänzt werden.
\end{description}

\subsection*{Spielbrett}

% Board
\item Erstellen Sie eine Spielbrett-Klasse mit folgenden Merkmalen:
\begin{itemize}
\item Implementieren Sie das Interface \code{flowerwarspp.preset.Board}.
\item Ein Spielfeld der Größe $n$ mit $3 \le n \le 30$.
\item Es muss einen Konstruktor geben, dem nur die Spielfeldgröße als \code{int} übergeben wird. Das wird benötigt, um das Spielbrett automatisiert testen zu können.

\underline{Hinweis}

Ob Sie alles richtig gemacht haben können Sie leicht testen, indem Sie Ihre Implementation des Spielbretts testen lassen.
\item Gültige Spielzüge und Spielzüge die zum Ende des Spiels führen werden erkannt.
\item Der erste entgegengenommene Spielzug gehört immer zum roten Spieler.
\item Ein Spielzug ist ein Objekt der Klasse \code{flowerwarspp.preset.Move}. Ein Move hat immer eines der nachfolgenden Formate:
\begin{itemize}
\item Zwei Referenzen auf Objekte der Klasse \code{flowerwarspp.preset.Flower}
\item Eine Referenz auf ein Objekt der Klasse \code{flowerwarspp.preset.Ditch}
\item Keine Referenzen, dafür aber den Typ \code{flowerwarspp.preset.MoveType} mit Wert \code{End}
\item Keine Referenzen, dafür aber den Typ \code{flowerwarspp.preset.MoveType} mit Wert \code{Surrender}
\end{itemize}
\item Die Schnittstelle \code{towerwarspp.preset.Viewable} wird implementiert. 
\end{itemize}

\subsection*{Ein- und Ausgabe}
% Viewer
\item Erstellen Sie eine Klasse, die die Schnittstelle \code{flowerwarspp.preset.Viewer} implementiert.

Diese Klasse soll es ermöglichen alle für das Anzeigen eines Spielbrett-Objekts nötigen Informationen zu erfragen, ohne Zugriff auf die Attribute des Spielbrett-Objekts zuzulassen.

Die Methode \code{viewer()} des Spielbretts liefert ein passendes Objekt dieser Klasse. Aus diesem Grund muss die Spielbrettklasse alle notwendige Funktionalität enthalten, um die Funktionen des \code{flowerwarspp.preset.Viewer} Interfaces umsetzen zu können.

% Text Input
\item Erstellen Sie eine Text-Eingabe-Klasse, die die Schnittstelle \\ \code{towerwarspp.preset.Requestable} implementiert.

Die Methode \code{request()} fordert einen Zug, in einer Zeile, von der Standardeingabe an und liefert ein dazu passendes \code{flowerwarspp.preset.Move}-Objekt zurück.

Verwenden Sie die statische Methode \code{parseMove(String)} der Klasse \code{Move}, um den von der Standardeingabe eingelesenen String in ein Move-Objekt umzuwandeln.

Die Methode \code{parseMove} wirft eine \code{flowerwarspp.preset.MoveFormatException}, falls das Einlesen missglücken sollte. Auf diese Exception muss sinnvoll reagiert werden.

% Graphic
\item Entwerfen Sie eine Schnittstelle für die Ausgabe des Spielbretts. Verwenden Sie die \code{flowerwarspp.preset.Viewer} Schnittstelle, um Informationen über das Spielbrett an die anzeigende Klasse weiterzugeben.

\underline{Hinweis}

Sie können sich hierfür an der Klasse \code{flowerwarspp.boarddisplay.BoardDisplay} orientieren, die Ihnen auf der GitLab Seite zur Verfügung steht.

\item Erstellen Sie eine grafische Ein-Ausgabe-Klasse. Diese Klasse implementiert die Schnittstellen \code{flowerwarspp.preset.Requestable} und die von Ihnen geschriebene Ausgabe-Schnittstelle und benutzt ein Objekt einer Klasse, die die Schnittstelle \code{flowerwarspp.preset.Viewer} implementiert, für eine einfache grafische Ausgabe.

Sorgen Sie dafür, dass die Darstellung des Spielbretts der Größe des Fensters angepasst ist und beim Verändern der Fenstergröße mitskaliert. Alle Informationen zum Status des Spiels müssen auf der grafischen Ausgabe erkennbar sein (gepflanzte Blumen, gebaute Gräben, Punktestand, Sieger bei Spielende)

Sorgen Sie dafür, dass von der grafischen Eingabe nur gültige Züge zurückgeliefert werden.

\underline{Hinweis}

Investieren Sie nicht zu viel Zeit in das Design, denn es wird nicht bewertet.
\newpage

\subsection*{Spieler}

% Player
\item Alle Spieler implementieren die Schnittstelle \code{flowerwarspp.preset.Player}.

\begin{itemize}
\item Ein Spieler hat keine Referenz auf das Spielbrett-Objekt des Programmteils, der die Züge anfordert. Trotzdem muss ein Spieler den Spielverlauf dokumentieren, damit er gültige Züge identifizieren kann. Dazu erzeugt jeder Spieler ein eigenes Spielbrett-Objekt und setzt seine und die Züge des Gegenspielers auf diesem Brett.

Daraus können sich Widersprüche zwischen dem Status des eigenen Spielbretts und dem gelieferten Status des Spielbretts des Hauptprogramms ergeben. Das ist ein Fehler auf den mit einer Exception reagiert wird.

\item Die Methoden der Player-Schnittstelle müssen in der richtigen Reihenfolge aufgerufen werden. Eine Abweichung davon ist ein Fehler auf den mit einer Exception reagiert werden muss.

Ein Spieler wird zu Spielbeginn mit einem Aufruf von \code{init} initialisiert und durchläuft danach die Methoden \code{request}, \code{confirm} und danach \code{update} bis das Spiel endet. Im Falle eines blauen Spielers beginnt der Spieler mit \code{update} statt \code{request}. Der Zeitpunkt des Spielbeginns und eines erneuten Spiels ist für den Spieler nicht ersichtlich, \code{init} kann zu einem beliebigen Zeitpunkt aufgerufen werden.

\item Für ein problemloses Netzwerkspiel ist es nötig, dass die Spielerklassen nur \code{Exception}'s werfen und keine selbst erstellten Klassen, die von dieser erben. An jeder anderen Stelle im Spiel können eigene Exceptions frei erzeugt und geworfen werden.
\end{itemize}

Die Methoden dieser Schnittstelle sind wie folgt zu verstehen:

% Description player methods
\begin{itemize}[leftmargin=4em]
\item[\code{init}] \hfill \\Initialisiert den Spieler, sodass mit diesem Spieler-Objekt ein neues Spiel mit einem Spielbrett der Größe \code{size} und der durch den Parameter \code{color} bestimmten Farbe, begonnen werden kann.

Die Spielerfarbe ist einer der beiden Werte der Enumeration \\
\code{flowerwarspp.preset.PlayerColor} und kann die Werte \code{Red} und \code{Blue} annehmen.
\item[\code{request}] \hfill \\Fordert vom Spieler einen Zug an.
\item[\code{confirm}] \hfill \\Übergibt dem Spieler im Parameter \code{status} Informationen über den letzten mit \code{request} vom Spieler gelieferten Zug.

\underline{Beispiele}
\begin{itemize}
\item Gilt \code{status == }\textit{eigener Status} und\dots
\begin{itemize}
\item \dots \code{status == Status.Ok} war der letzte Zug gültig
\item \dots \code{status == Status.Red\_Win} war der letzte Zug gültig und der rote Spieler hat das Spiel gewonnen
\end{itemize}
\item Gilt \code{status != }\textit{eigener Status} stimmt der Status nicht mit dem spielereigenen Spielbrett überein. Hier muss mit einer entsprechenden Exception reagiert werden!
\end{itemize}
\item[\code{update}] \hfill \\Liefert dem Spieler im Parameter \code{opponentMove} den letzten Zug des Gegners und im Parameter \code{status} Informationen über diesen Zug.

\underline{Hinweis}

Hier gelten die gleichen Beispiele wie auch für \code{confirm}.
\end{itemize}

% Human Player
\item Erstellen Sie eine Interaktive-Spieler-Klasse, die die Schnittstelle \\ \code{flowerwarspp.preset.Player} implementiert.

Ein Interaktiver-Spieler benutzt ein Objekt einer Klasse, die das Interface \\ \code{flowerwarspp.preset.Requestable} implementiert, um Züge vom Benutzer anzufordern.

% Random AI
\item Erstellen Sie eine Computerspieler-Klasse, die die Spieler-Schnittstelle implementiert und gültige, aber nicht notwendigerweise zielgerichtete, Züge generiert. Dazu wird aus allen aktuell möglichen gültigen Spielzügen zufällig ein Zug ausgewählt.

\underline{Hinweis}

Java stellt für die Erzeugung von Pseudozufallszahlen die Klasse \code{java.util.Random} zur Verfügung.

% Simple AI
\item Erstellen Sie einen weiteren Computerspieler, der zielgerichtet, entsprechend der einfachen Strategie, versucht das Spiel zu gewinnen.

Die Strategie ist auf der GitLab Seite beschrieben: \\
\href{https://gitlab.gwdg.de/app/flowerwarspp/blob/master/specification/simple-strategy.md}{https://gitlab.gwdg.de/app/flowerwarspp/blob/master/specification/simple-strategy.md}

% Network
\item Programmieren Sie einen Netzwerkspieler mit dem sie jede Implementation der Schnittstelle \code{flowerwarspp.preset.Player} einer anderen FlowerWarsPP- \\Implementation anbieten können.

Falls Sie den Netzwerkspieler im Netzwerk anbieten möchten, läuft die Spiellogik auf einer entfernten FlowerWarsPP-Implementation. Sehen Sie hierfür eine Möglichkeit vor, das Spiel dennoch über die selbst geschriebene Ausgabe-Schnittstelle zu verfolgen.

% Beispiel zum Umgang mit RMI habe ich absichtlich weggelassen.. In der Vorlesung werden ja bereits Beispiele mitgegeben und den Gruppen soll hier vielleicht etwas Raum zur eigenen Arbeit gelassen werden.

\subsection*{Hauptprogramm}
% Run class
\item Erstellen Sie eine ausführbare Klasse mit folgender Funktionalität.
\begin{itemize}
\item Die Auswahl der roten und blauen Spieler Klassen (Interaktiver Spieler, einer der Computerspieler) und die Größe des Spielbretts soll beim Starten des Programms über die Kommandozeile festgelegt werden können.

Verwenden Sie zum Einlesen der Kommandozeilenparameter und zum Abfragen der entsprechenden Einstellungen ein Objekt der Klasse \\
\code{flowerwarspp.preset.ArgumentParser}.

Auf der GitLab Seite sind nähere Informationen zum Umgang mit dieser Klasse zu finden.

\item Ein Spielbrett in Ausgangsposition mit der eingestellten Größe wird initialisiert.
\item Zwei Spielerobjekte werden wie eingestellt erzeugt und über Referenzen der \code{flowerwarspp.preset.Player}-Schnittstelle angesprochen. 

Beide Spieler benutzen dasselbe Objekt einer Klasse, die das \code{Requestable}-Interface implementiert, um Züge vom Benutzer anzufordern.
\item Von den Spieler-Referenzen werden abwechselnd Züge erfragt. Gültige Züge werden bestätigt und dem jeweils anderen Spieler mitgeteilt.
\item Die gültigen Züge werden auf dem Spielbrett ausgeführt.
\item Der aktuelle Stand des Spiels (und des Spielbretts) wird über die selbst geschriebene Ausgabe-Schnittstelle ausgegeben.
\item Wenn ein Zug zum Spielende führt, macht die Ausgabe eine Meldung darüber.
\item Sorgen Sie dafür, dass man das Spiel Computer gegen Computer gut verfolgen kann, verwenden Sie hierfür den Kommandozeilenparameter \code{-delay}.
\item Sehen Sie eine Möglichkeit vor über das Netzwerk zu spielen. 

Ein Netzwerkspiel findet statt, wenn mindestens einer der Spieler den Typ \code{REMOTE} hat (siehe \code{flowerwarspp.preset.PlayerType}) oder wenn ein Spieler im Netzwerk angeboten wird.

Das Hauptspiel behandelt einen Netzwerkspieler über die Schnittstelle\\ \code{flowerwarspp.preset.Player} wie jeden anderen Spieler auch.

Sehen Sie im Falle eines \code{REMOTE}-Spielers eine Möglichkeit vor, diesen zu finden (Name, Host, Port). Dies können Sie zum Beispiel über weitere Kommandozeilenparameter steuern oder interaktiv abfragen.

Wenn Sie einen Netzwerkspieler anbieten möchten, wählen Sie auch hier eine geeignete Methode den Spielertypen und den Namen einzustellen, unter dem der Spieler an der RMI Registry registriert werden soll.

Beim Anbieten wird keine Farbe festgelegt, da der Spieler diese Information beim Aufruf von \code{init} mitgeteilt bekommt.
\end{itemize}

% Ant
\item Verwenden Sie \emph{Ant} zum automatisierten Übersetzen des Programms und zum Erstellen der Dokumentation.

\item \textbf{Optional.} Bauen Sie das Spiel weiter aus.
\begin{itemize}
\item Laden und Speichern von Spielständen
\item Implementieren Sie einen Turniermodus
\item Erstellen Sie einen weiteren, intelligenteren Computerspieler, z.B. durch die Vorrausberechnung weiterer Züge und/oder einer besser balancierten und/oder erweiterten Bewertung.
\item Erweitern Sie die grafische Ein-Ausgabe-Klasse um mehr Funktionalität (Anzeigen von gültigen Zügen / Anzeige von Feldern, die nicht mehr bepflanzt werden können)
\item \dots
\end{itemize}
\end{enumerate}
 \newpage
\section*{Fragen und Antworten}
Falls Sie Fragen zu den Spielregeln haben oder Sie vor einer Spielsituation stehen, die mit den Spielregeln nicht eindeutig geregelt sind, nutzen Sie bitte das \textbf{Issue-System} von GitLab.

Bei Bedarf werden dann die Regeln um weitere Beispiele erweitert oder ergänzende Informationen zur Verfügung gestellt.

\section*{Anforderungen an das fertige Projekt}
\begin{enumerate}
\item Per E-Mail an \code{brosenne@informatik.uni-goettingen.de} wird eine Anleitung und ein Archiv (tar, zip, etc.) ausgeliefert.
\item Das Archiv enthält den Quelltext des FlowerWarsPP-Computerspiels, der sich im Rechnerpool des Instituts für Informatik übersetzen und starten lässt.

Es gibt ein Ant-Buildfile, das eine lauffähige Version des Spiels, gepackt in ein Jar-File, und die vollständige API-Dokumentation erzeugen kann.
\item Die Anleitung beschreibt wie das Archiv zu entpacken ist, der Quelltext übersetzt, die API-Dokumentation erzeugt und das FlowerWarsPP-Computerspiel gestartet wird. Weiterhin wird die Bedienung Ihrer Spielimplementation beschrieben. Geben Sie diese Anleitung im PDF-Format oder als GitLab Flavored Markdown (GFM) ab.

\underline{Hinweis}

Sie müssen in der Anleitung nicht die Spielregeln erneut erklären. Es geht um die Nutzung Ihrer konkreten Implementation!
\end{enumerate}


\end{document}
